% Awesome CV LaTeX Template
%
% This template has been downloaded from:
% https://github.com/huajh/huajh-awesome-latex-cv
%
% Author:
% Junhao Hua


%Section: Work Experience at the top
\sectionTitle{实习/项目经历}{\faCode}

\begin{experiences}

 \experience
    {2019年4月至今}   {多触点归因模型}{ Multi-touch attribution}{京东广告数据部}
     {}{
                      \begin{itemize}

                        \item 代码重构。将串行业务逻辑拆分为并行结构 \textbf{->}业务时间缩短30\%,减小代码的耦合度,提高代码复用性;
                        \item 改进特征类型转换方式。将原来使用Python进行Parquet到TFRecord转换,改进为使用Spark进行分布式转换\textbf{->}将转换时间缩短60\%;
                        \item 增加曝光模型。原项目主要针对用户的点击进行广告位归因,增加了基于广告位曝光的模型;

                      \end{itemize}
                    }
                    {biRNN, Spark, Tensorflow-Keras, Scala,Python}
  \emptySeparator
  \experience
    { } {广告主预算分配}{Budget Allocation}{京东广告数据部}
    {}    {
                      \begin{itemize}
                        \item 根据原始技术文档编写初始版本的模型系统,并进行模拟测试;
                        \item 改进初始模型的ROI(投资回报率)预测的线性回归模型。使用Xgboost算法进行该子模型的改进\textbf{->}将\emph{陈天奇}开发的Xgboost-Spark编译,以免部署的方式运行在Spark3.0集群上,使得模型的\emph{rmse}下降了351。 \textbf{并进一步将ROI的预测从天级别提升到小时级别};
                        \item 使用Maven将Xgboost算法依赖到JD的依赖库中,并进行高层的封装和框架的设计。简化了组内复用Xgboost4Spark算法的成本,并在此基础上,先后封装了GDBT、LinearRegression等常用算法;
                        \item 改进初始模型的CAP(广告位花销能力)预测。使用内部AutoML框架将原有的基于规则的CAP预测升级为稳定可靠的模型; \textbf{正在进行CAP实时预测工作};
                        \item 针对SparkML对可视化支持比较差的问题,结合ClickHouse和DarshBoard搭建了模型结果可视化的Pipline;
                      \end{itemize}
                    }
                    {XGBoost,Word2Vector,LinearRegression,SparkML,SparkSQL,SparkStreaming,AutoML}
  \emptySeparator

  \experience
  {2017年10月} {风机叶片开裂预警}{CCF工业大数据竞赛}{实验室}
  {}    {
				  	\begin{itemize}
				  		\item  负责完成数据清洗工作,包括缺失值处理、规范化操作以及对数据不平衡问题的处理;
				  		\item  负责特征的选择工作,通过分析各个特征间的相关系数、方差、信息系数等来选择最终特征;
                        \item  负责部分模型的构建工作,包括编写LR算法完成模型的构建,通过网格搜索来进行模型的调参;
				  	\end{itemize}
				  }
				  {特征工程, 逻辑回归,网格搜索}
  \emptySeparator
  \experience
  {2018年9月} {工业控制网络异常行为识别}{国家电网}{实验室项目}
  {}    {
    \begin{itemize}
      \item 负责构建网络流量数据包获取,包括网络数据包深度解析,主要通过交错时间窗的方式统计流量数据包五元组;
      \item  负责构建语义特征, 使用 Skip-gram 将数据包转化为向量表示、 并通过 PV-DM 和 PV-DBOW 结合的方式建
  立行为特征向量;
  \item 负责模型的构建工作, 包括编写 OCSVM 算法完成模型构建, 通过模型调参选择最优参数;
    \end{itemize}
  }
  {Skip-gram, OCSVM}
  \emptySeparator
  \experience
  {2014年10月} {基于Android智能家居房间控制系统}{山东建筑大学}{山东建筑大学机器人与人工智能实验室}
  {}    {
				  	\begin{itemize}
				  		\item 参与了整个系统的设计;
				  		\item 负责智能调配中心的设计与开发工作;
                        \item 负责\textbf{基于WIFI的室内定位算法的设计和开发工作};
				  	\end{itemize}
				  }
				  {软件开发, 数据库, JAVA, Android}

 \emptySeparator
\end{experiences}
